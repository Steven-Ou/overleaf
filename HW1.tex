\documentclass[12pt]{article}

\include{preamble}
%All credit goes towards Dr. Adam Kapelner for providing the preamble and homework template.

\newtoggle{professormode}




\title{MATH 245 Spring \the\year~ Homework \#1}

\author{Steven Ou} %STUDENTS: write your name here

\iftoggle{professormode}{
\date{Due via Brightspace 11:59PM, February 20, \the\year \\ \vspace{0.5cm} \small (this document last updated \today ~at \currenttime)}
}

\renewcommand{\abstractname}{Instructions and Philosophy}

\begin{document}
\maketitle

\iftoggle{professormode}{
\begin{abstract}
The path to success in this class is to do many problems. Unlike other courses, exclusively doing reading(s) will not help. Coming to lecture is akin to watching workout videos; thinking about and solving problems on your own is the actual ``working out.''  Feel free to \qu{work out} with others; \textbf{I want you to work on this in groups.}

The problems below are color coded: \ingreen{green} problems are considered \textit{easy} and marked \qu{[easy]}; \inorange{yellow} problems are considered \textit{intermediate} and marked \qu{[harder]}, \inred{red} problems are considered \textit{difficult} and marked \qu{[difficult]} and \inpurple{purple} problems are extra credit. The \textit{easy} problems are intended to be ``giveaways'' if you went to class. Do as much as you can of the others; I expect you to at least attempt the \textit{difficult} problems. 

Bonus points are given as a bonus if the homework is typed using \LaTeX. Links to installing \LaTeX~and program for compiling \LaTeX~is found on the syllabus. You are encouraged to use \url{https://www.overleaf.com/}. If you are handing in homework this way, read the comments in the code; there is a line to comment out and you should replace my name with yours. The easiest way to use Overleaf is to go to \textit{Menu} and select \textit{Copy Project}. If you are asked to make drawings, you can take a picture of your handwritten drawing and insert them as figures or leave space using the \qu{$\backslash$vspace} command and draw them in after printing or attach them stapled.


\end{abstract}

\thispagestyle{empty}
\vspace{1cm}
NAME:  \line(1,0){380}
\clearpage
}

\problem{In this exercise, we will review some general ideas about modeling. }

\begin{enumerate}

\easysubproblem{ What is a \textit{model}?
}
\vspace{5mm}
\newline \newline A model is just an approximation to reality. 
\vspace{5mm}

\intermediatesubproblem{Give an example of a model (preferably one that was not discussed in class) and explain why this is a model.}
\vspace{5mm}
\newline \newline An example of a model that we did not discuss in class is a cart, which models a real-life car. Even though it doesn't perfectly simulate a car, a cart is considered a model since we can use it as a reference to an actual car. We can also use simulations to make predictions about certain aspects of actual cars, such as their speed.
\vspace{5mm}

\intermediatesubproblem{George Box stated that \qu{All models are wrong \ldots}. If all models are wrong, then why do we study models?}
\vspace{5mm}
\newline \newline Even if all models are wrong, we can still make predictions, test hypotheses, and improve decision-making, even though they are never perfect. Models can evolve and become more accurate and useful for scientific and experimental use. 

\vspace{5mm}
\easysubproblem{ What are the two goals of modeling? }
\newline \newline The two goals of modeling are prediction and explanation.  
\vspace{10mm}

\easysubproblem{Suppose there is some phenomenon that we wish to model. Do we usually know all of the inputs or drivers of this phenomenon? }
\newline \newline No, we don't know all the inputs or drivers of this phenomenon, because they may be difficult or impossible to measure accurately.
\vspace{5mm}

\easysubproblem{In class, we stated that finding a function or a model can be difficult and that recent advancements in artificial intelligence, particularly machine learning, can help us. What is artificial intelligence? }
\newline \newline Artificial intelligence is a term that refers to the general ability of computers to emulate human thoughts and use human intelligence to perform tasks. 
\vspace{5mm}

\easysubproblem{In class, we stated that finding a function or a model can be difficult and that recent advancements in artificial intelligence, particularly machine learning, can help us. What is machine learning? (Use the most recent definition we discussed in class). } 
\newline \newline
Machine learning is used to train a piece of software to make useful predictions to generate content from data and to learn and improve their performance over time. 
\vspace{5mm} 

\easysubproblem{Read the section called \textit{History} on this page: \url{https://en.wikipedia.org/wiki/Machine_learning}. What was one of the first machine learning models?}
\newline \newline 
The first machine learning model is called \textcolor{red}{'Computer Checkers'}. 
\vspace{5mm}

\easysubproblem{Read the section called \textit{History} on this page: \url{https://en.wikipedia.org/wiki/Machine_learning}. How did Tom Mitchell define machine learning?}
\newline \newline 
Tom Mitchell defines machine learning by saying that the program performance \textcolor{purple} {(p)} on task \textcolor{orange} {(t)} as measured by \textcolor{purple} {p}, improves with experience \textcolor{green}{(e)}. This also means that the machine is like a baby and needs to be taught step by step as the baby goes through its task. And improves and learns every time the baby makes mistakes.
\vspace{5mm}

\easysubproblem{True or False? All of artificial intelligence falls under machine learning.}
\newline \newline 
False!
\vspace{5mm}

\easysubproblem{True or False? All of machine learning falls under artificial intelligence.}
\newline \newline
True!

\end{enumerate}


\problem{These exercises will review the need for gradient descent.}


\begin{enumerate}

\easysubproblem{Suppose we have a bunch of data points $(x_1, y_1), (x_2, y_2), \ldots, (x_n, y_n),$ and we wish to use a linear model to predict the value of $y$. What parameters (or values) are we looking to find?}
\newline \newline 
The parameters that we're looking to find are the slope (m) and the intercept (b) in the equation $\hat{y}=mx+b.$
\vspace{5mm}

\easysubproblem{In order to find these parameters, we have to define a cost function. How many dimensions would be required to graph the cost function?}
\newline \newline 
Graphing the cost function would require at least three dimensions because it depends on m and b and the cost value (c). 

\vspace{5mm}

\easysubproblem{Once we have our cost function, what exactly are we looking to do with this function? What is our primary objective?}
\newline \newline
What we want to do with the cost function is find the values of m and b that yield the lowest cost. Our primary objective is to obtain the best-fitting model. 
\vspace{5mm}

\easysubproblem{Explain how a machine can use the cost function to learn better values for our parameters. A full answer would outline the algorithm being used for the computer/machine to learn.}
\newline \newline 
A machine can 

\vspace{5mm}


\intermediatesubproblem{Using your answer above, explain what it means when we say the machine is learning?}


\vspace{60mm}


\intermediatesubproblem{Now suppose we have a bunch of data points $(x_1, y_1), (x_2, y_2), \ldots (x_n, y_n),$ and we wish to use a quadratic model to predict the value of $y$. That is, we wish to find real numbers $a$, $b$, and $c$ such that $\hat{y_i} = a{x_i}^2 + b{x_i} + c.$ How many dimensions would be required to plot the cost function? Can we actually graph this function? If not, then is there a particular technique we can use?}


\vspace{50mm}


\end{enumerate}

\problem{In class, we performed the gradient descent algorithm on $f(x) = (x-5)^2$ by starting at $x=0$ and using the learning rate, $\alpha = 0.1$.}


\begin{enumerate}

\intermediatesubproblem{This time, perform the next three iterations on $f(x) = (x-5)^2$ by starting at $x=0$ and using the learning rate, $\alpha = 4$. Describe what happens.}

\newpage

\end{enumerate}

\problem{In this exercise, we will practice a one dimensional gradient descent. For all of the parts below, let $f(x) = x^4 + x^3 - 2x^2$.}


\begin{enumerate}

\easysubproblem{Sketch a graph of $f(x)$.}

\vspace{60mm}


\intermediatesubproblem{Graphically, where is the global minimum of the function? If needed, round your answer to two decimal places.}


\vspace{30mm}


\intermediatesubproblem{Now suppose we wish to find the minimum by using gradient descent. Starting at $x=1$, compute what happens over the next three iterations. You may take the learning rate to be $\alpha = 0.1$. }

\newpage 

\extracreditsubproblem{This optional question is marked extra credit since we have yet to implement this in class. Compute what happens after 500 iterations. You may use a calculator, Excel, or a programming language. Please explain what you did to get your answer. }


\vspace{50mm}


\intermediatesubproblem{If you had to guess, what value do you think our algorithm in Problem 4 Part $(c)$ will converge to?}


\vspace{30mm}


\intermediatesubproblem{Now suppose we change our starting point to  $x= -1$. Compute what happens over the next three iterations. You may take the learning rate to be $\alpha = 0.1$. }

\newpage
\intermediatesubproblem{If you had to guess, what value do you think our algorithm in Problem 4 Part $(f)$ will converge to?}

\vspace{30mm}

\intermediatesubproblem{Do you think that in using gradient descent the starting point matters?}

\end{enumerate}


\end{document}
