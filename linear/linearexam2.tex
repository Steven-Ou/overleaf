\documentclass[10pt]{article}
\usepackage[T1]{fontenc}
\usepackage{amsmath, amssymb, array} % For math, symbols, and array formatting
\usepackage[letterpaper, margin=0.45in]{geometry} % Minimized margins for max content
\usepackage{multicol} % Essential for two-column layout
\usepackage{enumitem} % For custom list spacing
\usepackage[dvipsnames]{xcolor} % For custom colors
\usepackage[most]{tcolorbox} % For styled boxes
\usepackage{titlesec} % For custom section titles
\usepackage{booktabs} % For better table rules (if needed)

% --- LAYOUT & FONT ---
\setlength{\columnsep}{10pt} % Space between columns
\setlist{nosep, topsep=2pt, parsep=0pt, leftmargin=*} % Extremely compact lists
\titlespacing*{\section}{0pt}{*1.5}{*1} % Tighter section spacing
\titleformat*{\section}{\large\bfseries\sffamily} % Section font
\pagestyle{empty} % No page numbers

% --- THEME COLORS ---
\definecolor{boxbg}{gray}{0.95}
\definecolor{myblue}{named}{Blue}
\definecolor{mygreen}{named}{OliveGreen}
\definecolor{myred}{named}{Red}
\definecolor{myyellow}{named}{Apricot}

% --- CUSTOM TCOLORBOX STYLES ---
% Definition Box (Green)
\newtcolorbox{defbox}{
    enhanced,
    colback=mygreen!5!white,
    colframe=mygreen!75!black,
    boxrule=0.5pt,
    fonttitle=\bfseries\sffamily,
    coltitle=white,
    colbacktitle=mygreen!80!black,
    attach boxed title to top left={yshift=-2mm, xshift=3mm},
    boxed title style={boxrule=0.5pt, frame code={}},
    title=Definition,
    sharp corners,
    breakable
}
% Theorem Box (Blue)
\newtcolorbox{thmbox}{
    enhanced,
    colback=myblue!5!white,
    colframe=myblue!75!black,
    boxrule=0.5pt,
    fonttitle=\bfseries\sffamily,
    coltitle=white,
    colbacktitle=myblue!80!black,
    attach boxed title to top left={yshift=-2mm, xshift=3mm},
    boxed title style={boxrule=0.5pt, frame code={}},
    title=Theorem,
    sharp corners,
    breakable
}
% Example Box (Red)
\newtcolorbox{exbox}{
    enhanced,
    colback=myred!5!white,
    colframe=myred!75!black,
    boxrule=0.5pt,
    fonttitle=\bfseries\sffamily,
    coltitle=white,
    colbacktitle=myred!80!black,
    attach boxed title to top left={yshift=-2mm, xshift=3mm},
    boxed title style={boxrule=0.5pt, frame code={}},
    title=Example,
    sharp corners,
    breakable
}

% --- TITLE ---
\title{\textbf{\Huge MATH 231: Cumulative Review Sheet (Ch 1-5)}}
\author{Based on Lay, Lay, McDonald (6th Ed.)}
\date{Fall 2025}

\begin{document}
\maketitle
\thispagestyle{empty}
\scriptsize 

% --- WRAP EVERYTHING IN ONE MULTICOLS ENVIRONMENT ---
\begin{multicols}{2} 

\section*{I. Systems and RREF (1.1, 1.2)}
\begin{defbox}
\begin{itemize}
    \item \textbf{System:} A collection of linear equations.
    \item \textbf{Consistent/Inconsistent:} Has $\ge 1$ solution / Has no solution.
    \item \textbf{RREF (Reduced Echelon Form):} Unique matrix where pivots $= 1$ are the only nonzero entry in their column.
    \item \textbf{Pivot Column/Position:} Location containing a leading $1$ in the RREF.
    \item \textbf{Basic/Free Variable:} Corresponds to a pivot/non-pivot column.
\end{itemize}
\end{defbox}

\begin{thmbox}
\textbf{Thm 1.2: Existence and Uniqueness}
\begin{itemize}
    \item \textbf{Existence (Consistency):} System is consistent $\iff [A \mid \mathbf{b}]$ has \textbf{NO pivot in the rightmost column}.
    \item \textbf{Uniqueness:} If consistent: Unique Sol. $\iff$ No free variables. Infinite Sol. $\iff$ At least one free variable.
\end{itemize}
\end{thmbox}

\begin{exbox}
\textbf{Pivot/Free Variables}
$$
\text{RREF: } \begin{bmatrix}
\mathbf{1} & 0 & 3 & 0 & 5 \\
0 & \mathbf{1} & 2 & 0 & -3 \\
0 & 0 & 0 & \mathbf{1} & 0
\end{bmatrix} \text{}
$$
\begin{itemize}
    \item \textbf{Basic Variables:} $x_1, x_2, x_4$ (Pivots in Cols 1, 2, 4).
    \item \textbf{Free Variable:} $x_3$ (Non-Pivot Col 3).
    \item \textbf{Consistency:} Rightmost column (col 5) is NOT a pivot, so consistent.
\end{itemize}
\end{exbox}

\section*{II. Vectors, Span, and $\mathbf{Ax=b}$ (1.3, 1.4)}

\begin{thmbox}
\textbf{Thm 1.3: Equivalence of Views}
The linear system, the vector equation ($x_1 \mathbf{a}_1 + \dots + x_n \mathbf{a}_n = \mathbf{b}$), and the matrix equation ($A\mathbf{x} = \mathbf{b}$) \textbf{all share the same solution set}.
$\mathbf{b}$ is in Span$\{\mathbf{a}_1, \dots, \mathbf{a}_n\}$ $\iff A\mathbf{x} = \mathbf{b}$ is consistent.

\textbf{Thm 1.4: Span $\mathbb{R}^m$ Equivalence}
Let $A$ be $m \times n$. The following are equivalent:
\begin{enumerate}
    \item For each $\mathbf{b} \in \mathbb{R}^m$, $A\mathbf{x} = \mathbf{b}$ has a solution.
    \item The columns of $A$ \textbf{span $\mathbb{R}^m$}.
    \item $A$ has a \textbf{pivot position in every row}.
\end{enumerate}
\end{thmbox}

\section*{III. Solution Sets Structure (1.5)}
\begin{defbox}
\textbf{Homogeneous $\mathbf{Ax=0}$}
\begin{itemize}
    \item Always consistent (trivial sol $\mathbf{x}=\mathbf{0}$).
    \item Has a \textbf{nontrivial solution} ($\mathbf{x} \ne \mathbf{0}$) $\iff$ there is \textbf{at least one free variable}.
\end{itemize}
\end{defbox}
\begin{thmbox}
\textbf{Thm 1.6: Non-Homogeneous Solution Set}
If $A\mathbf{x} = \mathbf{b}$ is consistent, the solution set is:
$$
\mathbf{x} = \mathbf{p} + \mathbf{v}_h \quad (\text{Particular} + \text{Homogeneous}) \text{}
$$
The set is a translation of the homogeneous solution space ($\text{Span}\{\mathbf{v}_1, \mathbf{v}_2, \dots\}$).
\end{thmbox}

\begin{exbox}
\textbf{Parametric Vector Form (PVF)}
\begin{itemize}
    \item $\text{Setup (from Ex. 1): } x_1 = 5 - 3x_3$, $x_2 = -3 - 2x_3$, $x_3=t$ (free), $x_4 = 0$.
\end{itemize}
General solution in PVF (Decomposition):
$$
\mathbf{x} = \begin{bmatrix} x_1 \\ x_2 \\ x_3 \\ x_4 \end{bmatrix} = \underbrace{\begin{bmatrix} 5 \\ -3 \\ 0 \\ 0 \end{bmatrix}}_{\mathbf{p}} + t \underbrace{\begin{bmatrix} -3 \\ -2 \\ 1 \\ 0 \end{bmatrix}}_{\mathbf{v}_h} \text{}
$$
\end{exbox}

\section*{IV. Linear Independence (1.7)}
\begin{defbox}
\begin{itemize}
    \item \textbf{Linearly Independent (LI):} $x_1 \mathbf{v}_1 + \dots + x_p \mathbf{v}_p = \mathbf{0}$ has \textbf{only the trivial solution}.
    \item \textbf{Tests for LD (Instant Dependence):}
    \begin{itemize}
        \item Contains the $\mathbf{0}$ vector.
        \item More vectors than entries ($p>n$ in $\mathbb{R}^n$).
        \item Matrix $A=[\mathbf{v}_1 \dots \mathbf{v}_p]$ has a \textbf{free variable}.
    \end{itemize}
\end{itemize}
\end{defbox}
\begin{thmbox}
\textbf{Thm 1.7: Characterization of LD}
A set $S$ is $\mathbf{LD} \iff$ at least one vector in $S$ is an \textbf{LC of the others}.
\end{thmbox}

\section*{V. Linear Transformations (1.8, 1.9)}
\begin{thmbox}
\textbf{Thm 1.10: The Standard Matrix $\mathbf{A}$}
$T(\mathbf{x}) = A\mathbf{x}$, where $A$ is the unique standard matrix formed by transforming the standard basis vectors:
$$
A = [T(\mathbf{e}_1) \ T(\mathbf{e}_2) \ \dots \ T(\mathbf{e}_n)] \text{}
$$
\textbf{Thm 1.11 \& 1.12: One-to-One and Onto}
Let $A$ be the $m \times n$ standard matrix for $T: \mathbb{R}^n \to \mathbb{R}^m$.
\begin{itemize}
    \item \textbf{T is Onto (Surjective) $\iff$} $A$ has a \textbf{pivot in every row}.
    \item \textbf{T is One-to-One (Injective) $\iff$} $A$ has a \textbf{pivot in every column}.
\end{itemize}
\end{thmbox}

\section*{VI. Matrix Algebra (2.1)}
\begin{defbox}
\textbf{Matrix Multiplication $\mathbf{AB}$}
\begin{itemize}
    \item Defined only if (\# cols in $A$) = (\# rows in $B$).
    \item $\text{Col}_j(AB) = A \cdot \text{Col}_j(B)$.
    \item \textbf{Warning:} $AB \ne BA$ generally.
\end{itemize}
\textbf{Transpose Properties}
\begin{itemize}
    \item $(A+B)^T = A^T + B^T$.
    \item $(AB)^T = B^T A^T$ (\textbf{Reversal Rule}).
\end{itemize}
\end{defbox}

\section*{VII. The Inverse (2.2, 2.3)}
\begin{defbox}
\textbf{2x2 Inverse}: For $A = \begin{bmatrix} a & b \\ c & d \end{bmatrix}$, if $ad-bc \ne 0$:
$$ A^{-1} = \frac{1}{ad-bc} \begin{bmatrix} d & -b \\ -c & a \end{bmatrix} \text{} $$
\textbf{Algorithm to find $\mathbf{A}^{-1}$}:
Row reduce the augmented matrix $[\mathbf{A} \mid \mathbf{I}_n]$.
If $A \sim \mathbf{I}_n$, then $[\mathbf{A} \mid \mathbf{I}_n] \sim [\mathbf{I}_n \mid \mathbf{A}^{-1}]$.
\end{defbox}
\begin{thmbox}
\textbf{Properties of Inverses}
\begin{itemize}
    \item $(A^{-1})^{-1} = A$
    \item $(AB)^{-1} = B^{-1}A^{-1}$ (Reversal Rule)
    \item $(A^T)^{-1} = (A^{-1})^T$
    \item (From Quiz 5) If $AB$ and $B$ are invertible, $A$ must be invertible. Proof: $A = (AB)B^{-1}$
\end{itemize}
\end{thmbox}

\section*{VIII. Subspaces \& Rank (2.8, 2.9)}
\begin{defbox}
A \textbf{subspace} $H$ must satisfy:
\begin{enumerate}
    \item $\mathbf{0}$ is in $H$.
    \item Closed under addition ($\mathbf{u}+\mathbf{v} \in H$).
    \item Closed under scalar mult ($c\mathbf{u} \in H$).
\end{enumerate}
\begin{itemize}
    \item \textbf{Col Space $\text{Col}(A)$:} Span of $A$'s columns. (Subspace of $\mathbb{R}^m$). \textbf{Basis:} The pivot columns of the \emph{original} matrix $A$.
    \item \textbf{Nul Space $\text{Nul}(A)$:} Set of solutions to $A\mathbf{x}=\mathbf{0}$. (Subspace of $\mathbb{R}^n$). \textbf{Basis:} The vectors from the PVF of the solution.
\end{itemize}
\end{defbox}
\begin{thmbox}
\textbf{Rank-Nullity Theorem}
For an $m \times n$ matrix $A$:
$$ \text{rank}(A) + \text{dim}(\text{Nul}(A)) = n \text{} $$
$$ (\text{\# pivot cols}) + (\text{\# free variables}) = (\text{\# total cols}) \text{} $$
\end{thmbox}

\section*{IX. Determinants (3.1, 3.2)}
\begin{defbox}
\textbf{Cofactor Expansion} (across row $i$):
$\det(A) = \sum_{j=1}^n a_{ij} C_{ij}$, where $C_{ij} = (-1)^{i+j} \det(A_{ij})$
\begin{itemize}
    \item \textbf{Example (across row 1 for 3x3)}:
    $\det(A) = a_{11}\det(A_{11}) - a_{12}\det(A_{12}) + a_{13}\det(A_{13})$
\end{itemize}
\end{defbox}
\begin{thmbox}
\textbf{Properties \& Row Ops}
\begin{itemize}
    \item $\det(A)$ of a triangular matrix = product of its diagonal.
    \item \textbf{Swap} ($R_i \leftrightarrow R_j$): Multiplies det by -1.
    \item \textbf{Replacement} ($R_i + kR_j \to R_i$): \textbf{No change}.
    \item \textbf{Scaling} ($kR_i \to R_i$): Multiplies det by $k$.
    \item $A$ is invertible $\iff \det(A) \neq 0$.
    \item $\det(AB) = \det(A) \det(B)$.
    \item $\det(A^T) = \det(A)$.
\end{itemize}
\end{thmbox}

% --- NEW SECTION ADDED HERE ---
\section*{X. Eigenvalues \& Eigenvectors (5.1, 5.2)}
\begin{defbox}
\begin{itemize}
    \item \textbf{Eigenvector:} A non-zero vector $\mathbf{x}$ such that $A\mathbf{x} = \lambda\mathbf{x}$ for some scalar $\lambda$.
    \item \textbf{Eigenvalue:} The scalar $\lambda$ above.
    \item \textbf{Eigenspace:} The set of all solutions to $(A - \lambda I)\mathbf{x} = \mathbf{0}$. This is the $\text{Nul}(A - \lambda I)$.
\end{itemize}
\end{defbox}
\begin{thmbox}
\textbf{Finding Eigenvalues}
\begin{itemize}
    \item $\lambda$ is an eigenvalue $\iff (A - \lambda I)\mathbf{x} = \mathbf{0}$ has a non-trivial solution.
    \item By the IMT, this is true $\iff \det(A - \lambda I) = 0$.
    \item This is the \textbf{Characteristic Equation}.
    \item The eigenvalues of a \textbf{triangular matrix} are the entries on its main diagonal.
\end{itemize}
\end{thmbox}

% --- THE BIG ONE ---
\begin{tcolorbox}[
    enhanced,
    colback=myyellow!80!White,
    colframe=Orange!50!Black,
    boxrule=1pt,
    fonttitle=\bfseries\sffamily,
    coltitle=black,
    colbacktitle=myyellow!80!White,
    attach boxed title to top center={yshift=-2mm},
    boxed title style={boxrule=0.5pt, colframe=Orange!50!Black},
    title=THE INVERSE MATRIX THEOREM (CUMULATIVE),
    sharp corners,
    breakable
]
Let $A$ be a square $n \times n$ matrix.
The following statements are all equivalent (all true or all false).
\begin{enumerate}[label=(\alph*)]
    \item $A$ is an invertible matrix.
    \item $A$ is row equivalent to the $n \times n$ identity matrix $I_n$.
    \item $A$ has $n$ pivot positions.
    \item The equation $A\mathbf{x} = \mathbf{0}$ has only the trivial solution.
    \item The columns of $A$ form a linearly independent set.
    \item The linear transformation $\mathbf{x} \mapsto A\mathbf{x}$ is one-to-one.
    \item The equation $A\mathbf{x} = \mathbf{b}$ has at least one solution for each $\mathbf{b}$ in $\mathbb{R}^n$.
    \item The columns of $A$ span $\mathbb{R}^n$.
    \item The linear transformation $\mathbf{x} \mapsto A\mathbf{x}$ maps $\mathbb{R}^n$ onto $\mathbb{R}^n$.
    \item There is an $n \times n$ matrix $C$ such that $CA = I_n$.
    \item There is an $n \times n$ matrix $D$ such that $AD = I_n$.
    \item $A^T$ is an invertible matrix.
    \item The columns of $A$ form a basis for $\mathbb{R}^n$.
    \item $\text{Col}(A) = \mathbb{R}^n$.
    \item $\text{dim}(\text{Col}(A)) = n$.
    \item $\text{rank}(A) = n$.
    \item $\text{Nul}(A) = \{\mathbf{0}\}$.
    \item $\text{dim}(\text{Nul}(A)) = 0$.
    \item The number $0$ is \textbf{not} an eigenvalue of $A$.
    \item The determinant of $A$ is \textbf{not} zero ($\det(A) \neq 0$).
\end{enumerate}
\end{tcolorbox}
% Note: Eigenvalue statement (s) is from Ch 5,
% but is a key part of the IMT.

\end{multicols} % <<< END THE ONLY MULTICOLS ENVIRONMENT >>>
\end{document}